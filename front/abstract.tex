% !TeX root = ../main.tex

\begin{abstract}

近年來,神經網絡視訊編碼(Neural Video Compression, NVC)在壓縮效率上已展現出超越傳統編碼標準
(如 H.265/HEVC 與 H.266/VVC)的潛力。然而,現有的 NVC 研究多聚焦於網絡架構的率失真(Rate-Distortion, R-D)
性能優化,卻鮮少探討實際應用中至關重要的碼率控制(Rate Control)機制。由於 NVC 的編碼特性具有高度非線性且對量化參數
(Quantization Parameter, QP)的變動極為敏感,直接套用傳統的碼率控制演算法往往會導致 QP 劇烈震盪,
進而嚴重損害幀間預測的品質與整體的壓縮效率。

\end{abstract}

\begin{abstract*}

In recent years, Neural Video Compression (NVC) has demonstrated significant potential in 
surpassing traditional video coding standards, such as H.265/HEVC and H.266/VVC, in terms of compression efficiency. 
However, existing research on NVC predominantly focuses on optimizing Rate-Distortion (R-D) performance through 
architectural innovations, often overlooking the critical aspect of Rate Control (RC) required for practical applications. 
Due to the highly non-linear nature of NVC and its sensitivity to Quantization Parameter (QP) fluctuations, 
directly applying traditional rate control algorithms often leads to severe QP oscillations. 
These oscillations disrupt inter-frame prediction quality, thereby degrading the overall compression efficiency.

\end{abstract*}