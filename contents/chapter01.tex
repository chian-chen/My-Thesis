% !TeX root = ../main.tex

\chapter{Introduction}




\section{Background}
% 描述壓縮如何在生活中扮演重要角色,引用一些技術報告,目前網路上多少比例的
% 流量來自影片串流,不同的社群平台乃至現在的 AI 生成照片和影片,造成人們互動的變革,
% Nowadays, video content contributes to more than 80% internet traffic, 
% and the percentage is expected to increase even further. Therefore,
% it is critical to build an efficient video compression system and
% generate higher quality frames at given bandwidth budget
% 簡述 NVC 的崛起以及為何 Rate Control 在實際應用(如視訊會議、直播)中不可或缺。
% RC 的必要性: 闡明 Bitrate(位元率)是唯一的物理資源約束。在即時應用中,一旦位元率不穩定,
% 就會導致網路緩衝區溢出(Overflow)或不足(Underflow),直接影響使用者體驗。
% QoS: Quality of Service 概念: 要達到好的 Service,需要穩定且可靠的 Rate Control

Modern digital life and network communication are fundamentally built upon efficient data compression 
technologies. With the explosive growth of global data volume—particularly the rapid increase in multimedia 
content—compression techniques have become increasingly critical for maintaining Quality of Service (QoS).

Video content has now become the dominant source of network traffic. According to the Ericsson Mobility Report\cite{57}, 
video traffic is projected to account for approximately 76\% of all mobile data traffic by the end of 2025. 
This trend is driven not only by traditional video streaming platforms such as YouTube, Netflix, and TikTok, 
but has also been further intensified by the widespread adoption of short-form video sharing on social media, 
real-time video conferencing, and the growing prevalence of high-quality images and videos generated by 
artificial intelligence (AI). Consequently, designing high-performance video compression systems that 
can deliver higher visual quality under limited bandwidth budgets has become an increasingly urgent challenge.

Although conventional video coding standards—such as H.264/AVC\cite{43}, H.265/HEVC\cite{38}, and 
Versatile Video Coding (VVC)\cite{44}—have achieved remarkable success and have been continuously optimized over the 
past decades, Neural Video Coding (NVC) has emerged in recent years as a promising alternative. 
By leveraging the strong nonlinear modeling capability of neural networks, NVC has demonstrated significant 
potential to surpass traditional hybrid video coding frameworks in terms of compression efficiency \cite{61}.

With the rapid advancement of generative models in visual signal representation and processing, 
their impact has extended beyond academic research to international video coding standardization. 
Major standardization bodies, including the Joint Video Experts Team (JVET) and ISO/IEC MPEG, have 
actively explored neural network-based approaches to visual coding. A representative example is 
JPEG AI\cite{02}, a joint standardization effort between ISO/IEC JPEG and ITU-T, which adopts learning-based 
techniques for image compression while supporting downstream computer vision tasks. These efforts 
highlight neural video coding as a recognized and promising direction for future visual coding systems.




\section{Problem Statement}
% 傳統 RC 依賴於手動調整的 R-lambda 模型,這些模型無法準確捕捉 NVC 獨特的、隨內容而異的
% 非線性 (Non-linear) Rate-QP 特性,因此會導致 QP 震盪。
% DL 模型的局限: 現有基於 DL(如 RL)的 RC 方法,雖然理論上能處理非線性,但訓練成本高昂,
% 而且對於有即時時間要求的 Rate Control 算法,通常會造成額外的時間成本。
% 每個 NVC 架構不同,一但架構大改,Rate Control 的關係、模型如何達到都會重新變化,很難統一。

Despite the rapid progress of neural video coding (NVC), rate control (RC) remains a critical 
component for its practical and commercial deployment. In real-time applications such as video 
conferencing and live streaming, bitrate is the primary physical resource constraint, and unstable 
rate control may lead to buffer overflow or underflow, resulting in latency, frame drops, inefficient 
bandwidth utilization, and degraded Quality of Service (QoS). Therefore, a stable and reliable rate 
control mechanism is indispensable for real-time NVC systems.

Conventional rate control methods in hybrid video coding frameworks are typically derived from 
rate–distortion optimization (RDO) and rely on simplified and approximately stationary models that 
relate bitrate to the quantization parameter\cite{50}. Although effective for traditional codecs, these assumptions 
no longer hold for modern coding systems with increasingly flexible and adaptive coding structures.

In NVC, end-to-end learned representations introduce highly nonlinear, frame-level, and content-dependent 
rate characteristics, causing the Rate–QP relationship to become non-stationary and difficult to model. 
As a result, directly applying conventional model-based rate control methods often leads to unstable 
quantization parameter selection and inaccurate bitrate estimation in practical deployment.

In summary, there is currently a lack of real-time rate control solutions that simultaneously achieve 
high accuracy, low computational complexity, and robust adaptability to modern NVC architectures. 
Addressing this challenge is essential for enabling the practical deployment of neural video coding systems.

\section{Motivation}
% 目前 DCVC-RT [] 做到一個前所未有的里程碑(速度、效能兼具,完全超越 Traditional CODEC),
% 擺脫過去 NVC 仍然在速度上難以實用的印象。
% Rate Control 是 NVC 邁向實用的最後一哩路、但目前針對 NVC Model 進行的 Rate Control 研究還很少,
% 多數只針對 2024 以前的模型有零星的研究 []。DCVC-RT 足夠有代表性、
% 投入資源去研究目前最新 SOTA NVC 的 Rate-QP Relationship
% (另外,改善 CODEC 本身耗費算力巨大,需要超級大量的硬體資源,
% 目前 SOTA 大多集中在科技巨頭裡的 Research Team 像是 Microsoft, Ant Group 等等,這個大概不會寫進去)

Throughout the development of NVC, encoding and decoding speed has long been one of the primary bottlenecks 
limiting practical deployment. The emergence of DCVC-RT \cite{01}, however, marks an unprecedented milestone. 
DCVC-RT successfully achieves a favorable balance between computational efficiency and compression performance. 
In most scenarios, it surpasses conventional codecs in both speed and coding efficiency, and even outperforms 
experimental software codecs such as ECM\cite{ECM} that have yet to be officially released. 
This breakthrough effectively removes the long-standing perception that NVC is impractical due to 
excessive computational complexity, thereby enabling realistic prospects for commercial deployment.

Under this context, efficient rate control has become the “last mile” toward large-scale deployment of 
NVC systems. Although DCVC-RT has demonstrated the coding potential of modern NVC architectures, 
research on rate control specifically tailored to such state-of-the-art (SOTA) NVC models remains limited. 
Most existing studies \parencite{51, 52, 28} focus on earlier NVC architectures and rely on extensions of traditional mathematical models. 
Moreover, due to the rapid evolution of NVC architectures in recent years, it has been difficult to establish 
unified and generalizable experimental frameworks.

Given the superior performance and representative nature of DCVC-RT, investigating its Rate–QP 
relationship and designing a dedicated, lightweight rate control mechanism offers substantial academic and 
industrial value. Accordingly, this thesis focuses on DCVC-RT as a practically viable NVC model and aims to 
address the core rate control challenges that hinder the real-world deployment of neural video coding systems.


\section{Primary Contributions}
% 1. 提出一個自適應狀態空間模型 (Adaptive State-Space Model),
% 用於描述最新 NVC (DCVC-RT) 的非線性 Rate-QP 關係。並保持 Low-Complexity 
% 不影響 Real-time 模型進行壓縮的狀態。
% 首次將 RLS (遞迴最小平方法) 引入 Rate Control 模型控制,實現免梯度 (Gradient-free) 
% 且高精準度的即時位元率估計與控制。
% 2. 和過往傳統的 RC 控制和針對較久 NVC 設計的 RC 算法比起來,達到更佳的平衡表現。所提出的 RLS 
% 系統具有極低的運算負擔(約 O(N^2) 複雜度),滿足即時視訊編碼對低延遲的需求。
% 3. 對於各個算法和參數造成的影響,有比較全面的頗析,也補足目前發表的論文中普遍曖昧不清的初始化、
% 實驗設計等細節,對參數影響有教全盤的研究。

This thesis aims to address the aforementioned challenges by proposing an efficient and low-complexity 
real-time rate control framework for modern NVC architectures. The main contributions are summarized as follows:

\begin{enumerate}

\item \textbf{Adaptive State-Space Modeling for NVC Rate Control}

\begin{itemize}
\item This work proposes a low-complexity adaptive state-space model specifically designed to describe and 
predict the nonlinear Rate–QP relationship exhibited by modern NVC architectures such as DCVC-RT.
\item To the best of our knowledge, this study is the first to introduce Recursive Least Squares (RLS)\cite{62} into the 
rate control modeling of NVC systems, enabling gradient-free and highly accurate real-time bitrate estimation 
and control.
\end{itemize}

\item \textbf{Improved Performance–Complexity Trade-off}

\begin{itemize}
\item Compared with conventional rate control methods and algorithms designed for earlier NVC architectures, 
the proposed RLS-based system achieves a more favorable balance among bitrate accuracy, 
visual quality stability, and computational efficiency.
\item The proposed RLS framework incurs very low computational overhead, with approximately 
$O(N^2)$ complexity, where $N$ denotes the number of parameters in the state-space model. 
This fully satisfies the low-latency requirements of real-time video encoding.
\end{itemize}

\item \textbf{Comprehensive Parameter Analysis and Experimental Transparency}
\begin{itemize}
    \item This study provides a comprehensive analysis of the core algorithms and key parameters within the 
    proposed rate control framework.
    \item Experimental details that are often insufficiently documented in existing NVC literature—such as 
initialization strategies and experimental configurations—are explicitly discussed. The impact of these 
parameters is systematically evaluated, providing more reliable baselines and references for future research 
on NVC rate control.
\end{itemize}

\end{enumerate}

\section{Thesis Organization}

The remainder of this thesis is organized as follows. 
Chapter~2 introduces the fundamentals of video compression, including traditional hybrid coding frameworks 
and neural video compression (NVC). 
Chapter~3 reviews related work on state-of-the-art NVC architectures and existing rate control methods, 
and discusses their limitations. 
Chapter~4 presents the proposed adaptive state-space rate control framework based on recursive least squares (RLS). 
Chapter~5 evaluates the proposed method through experimental comparisons and ablation studies. 
Finally, Chapter~6 concludes the thesis and outlines directions for future work.
